%\documentclass[a4paper,11pt,twocolumn]{article}
\documentclass[twocolumn,aps,pre,groupedaddress]{revtex4-1}

\usepackage{amsmath}
\usepackage{graphicx}
\usepackage{amsmath}
\usepackage{graphicx}
\usepackage{subfig}
\usepackage{epstopdf}
\usepackage{amssymb}

\begin{document}
\title{Time Series Leanings}
\author{James M. McCracken}
\email{jmccrac1@masonlive.gmu.edu}
\affiliation{School of Physics, Astronomy, and Computational Sciences \\ George Mason University \\ 4400 University Drive MS 3F3, Fairfax,VA. 22030-4444}
\author{Robert S. Weigel}
\email{rweigel@gmu.edu}
\affiliation{School of Physics, Astronomy, and Computational Sciences \\ George Mason University \\ 4400 University Drive MS 3F3, Fairfax,VA. 22030-4444}
\date{\today}

\begin{abstract}
\end{abstract}

\pacs{}
\maketitle

\section{Introduction}

\section{Penchant Derivation}

\section{Mean Observed Leaning}

\subsection{Algorithm}

\section{Simple Example Systems}

\subsection{Impulse with Noisy Response Linear Example}
Consider the linear example dynamical system of
\begin{eqnarray}
\label{eq:linearex1}
X_t &=& \{0,2,0,0,2,0,0,2,0,0\}\\
Y_t &=& X_{t-1}+B\eta_t,
\end{eqnarray}
with $B\in\mathbb{R}\ge 0$ and $\eta_t\sim\mathcal{N}\left(0,1\right)$.  Specifically, consider $B\in[0,2]$ in increments of 0.02.  The response system $Y$ is just a lagged version of the driving signal with varying levels of standard Gaussian noise applied at each time step.  

\subsection{Cyclic Linear Example}
Consider the linear example dynamical system of
\begin{eqnarray}
\label{eq:linearex}
X_t &=& \sin(t)\\
Y_t &=& X_{t-1}+B\eta_t,
\end{eqnarray}
with $B\in\mathbb{R}\ge 0$ and $\eta_t\sim\mathcal{N}\left(0,1\right)$.  Specifically, consider $B\in[0,2]$ in increments of 0.02.  The response system $Y$ is just a lagged version of the driving signal with varying levels of standard Gaussian noise applied at each time step.  

\subsection{Non-Linear Example}
Consider the non-linear dynamical system of
\begin{eqnarray}
\label{eqn:nonlinearEX}
X_t &=& \sin(t)\\
Y_t &=& AX_{t-1}\left(1-BX_{t-1}\right)+C\eta_t,
\end{eqnarray}
with $A,B,C\in\mathbb{R}\ge 0$ and $\eta_t\sim\mathcal{N}\left(0,1\right)$.  Specifically, consider $A,B,C\in[0,5]$ in increments of 0.5.  

\subsection{RL Circuit Example}
\label{sec:rlcirc}
Both of the previous examples included a noise term, $\eta_t$.  Consider a series circuit containing a resistor, inductor, and time varying voltage source related by
\begin{equation}
\label{eqn:it}
\frac{dI}{dt} = \frac{V(t)}{L} - \frac{R}{L} I,
\end{equation}
where $I$ is the current at time $t$, $V(t)= \sin\left(\Omega t\right)$ is the voltage at time $t$, $R$ is the resistance, and $L$ is the inductance.  Eqn. \ref{eqn:it} was solved using the {\em ode45} integration function in MATLAB.  The time series $V(t)$ is created by defining values at fixed points and using linear interpolation to find the time steps required by the ODE solver.  

Consider the situation where $L=10$ Henries and $R=5$ Ohms are constant.  Physical intuition is that $V$ drives $I$, and so we expect to find that $V$ CCM causes $I$ (i.e., $C_{VI}>C_{IV}$ or $\Delta = C_{VI}-C_{IV} > 0$). 


\section{Empirical Data}

\section{Conclusion}

%\bibliographystyle{plain}
%\bibliography{main}

\end{document}
