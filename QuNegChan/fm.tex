
% abbreviate Sample FrontMatter

% we can take care of the frontmatter for you 

% we will need an Abstract and Keywords - please see a sample below

\def\TITLE{Negative Quantum Channels}
\def\SUBTITLE{{\normalsize An Introduction to Quantum Maps that Are Not Complete Positivity}}
\def\AUTHOR{James M. McCracken}
\def\AFFILIATION{George Mason University}
\def\LECTURE{\ \#1}
\def\lcSYNTHESIS{Synthesis Lectures on SAMPLE SERIES}
\def\SYNTHESIS{\uppercase{Synthesis Lectures on SAMPLE SERIES}}

%%%%%%%%%%%% FULL TITLE 5

\thispagestyle{emptyrule}
\title{\TITLE \\ \SUBTITLE}

\vspace*{2pc}
\authorname{\AUTHOR}
\authoraffiliation{\AFFILIATION}

\vfill
\synthesis{\SYNTHESIS\LECTURE}
\morganlogo
\clearpage


%%%%%%%%%%%% ABSTRACT AND KEY TERMS ( \keywords{ ... } ) 6

\thispagestyle{emptyrule}

\ABSTRACT
\noindent
This work is a brief introduction to negative quantum channels, i.e.\ linear, trace-preserving (and consistent) quantum maps that are not completely positive.  The flat and sharp operators are introduced and explained.  Complete positivity is presented as a mathematical property, but it is argued that complete positivity is not a physical requirement of all quantum operations.  Negativity, a measure of the lack of complete positivity, is proposed as a tool for empircally testing complete positivity assumptions.

\noindent

%end ABSTRACT

\keywords{%
complete positivity, quantum process tomography, quantum information, quantum channels, open quantum systems\\
}

\vfill
\pagebreak

\tableofcontents

\newpage
\thispagestyle{emptyrule}
\begin{center}
{\em Dedicated to Angel and G for their patience, understanding, and sense of humor---sorry this {\large \texttt{\textsc{jinkies}}} happened.}
\end{center}

\newpage
\chapter*{Preface}
The study of complete positivity in quantum operations has an interesting history.  The timeline seems to start in the mid-1950s \cite{Stinespring1955}, though it was not until the late 1960s that it started to get significant attention from physicists, primarly through the work of Sudarshan, Kraus\footnote{Kraus' most often cited work, \cite{Kraus1983}, is from 1983, but his publication record on this topic stretches back into the 1960s, e.g.\ \cite{Kraus1971} and \cite{Kraus1969}.} and others.  By the early 1970s, the seminal works of Sudarshan \cite{Sudarshan1976}, Lindblad \cite{Lindblad1976}, and Choi \cite{Choi1975} helped cement the usefulness of the complete positivty assumption in dealing with open quantum systems.  

All assumptions limit the theories that rely on them\footnote{Assuming a frictionless surface in a model of a wooden block sliding down a wooden inclined plane can work well until, for example, you need to explain why the bottom of the block is heating up.}, and the limiting nature of the complete positivity assumption was recognized very early on.  Sudarshan pointed it out in 1978 \cite{Sudarshan1978}.  Kraus discusses the dangers of applying his assumptions outside of ``scattering-type'' experiments in one of the sources most often cited in support of the complete positivity assumption \cite{Kraus1983}.  The early 1990s saw an article/response exchange between Pechukas and Alicki that raised awareness of the issue but did little to settle the argument of whether or not the assumption is required \cite{Pechukas1994,Alicki1995,Pechukas1995}.  Modern attitudes towards the assumption seem to be quite varied--from statements of the assumption being physically required \cite{Benatti2005,Alicki2001} to statements of the exact opposite \cite{Sudarshan2005}. 

Current attitudes towards this topic are nicely summed up in some of the comments I have received during the drafting of this work.  It was pointed out to me that the use of positivity domains as an alternative to the ``total domain argument for complete positivity'' (see Section \ref{sec:totaldomain}) might not be ``logically and physically more consistent than requesting complete positivity.''  A different reviewer (who was kind enough to reveiw my work multiple times) took a strong stance: ``Though it is not an ideological discussion about metaphysics, I understand it may appear such and mine a preconceived position. $\ldots$the author states that ``the assumption (of complete positivity) is not empirically justified.'' As I explained in my previous report, there is plenty of convincing arguments in the literature that, on the contrary, complete positivity is a physically necessary constraint because of the presence of quantum entanglement. It is true that complete positivity imposes constraints, but attributing them to unnecessary mathematical niceties is wrong.  $\ldots$ avoiding these constraints by limiting the action of a preparation map $\ldots$ is an ad hoc escamotage.  Furthermore, the physical evidence of negative maps should be accompanied by the physical evidence of negative probabilities which is exactly what complete positivity avoids.''  Another reviewer sums things up nicely with ``I am sympathetic to the author's goal of convincing others that the CP assumption is not physically appropriate, but $\ldots$ there is clear-cut evidence supporting that conclusion that is missing from the review.''  The ``clear-cut evidence'' mentioned by the third reviewer is a reference to the experimental oberservations of negative channels in \cite{Boulant2004}.  It would seem that the ``clear-cut'', decisive, evidence provided by the third reviewer is not possible according to the second reviewer.  Clearly, complete positivity is still a topic of passionate debate in the field.

My personal attitude towards complete positivity is, hopefully, clear from this manuscript.  I do, however, want to stress that this work is meant as an {\em introduction}.  A full book-length treatment of this topic would take several hundred pages and would, given the current rapid pace of the field, probably be out-of-date rather quickly.  As such, many relevant topics (and references) are nowhere to be found in the text that follows.  These excluded topics are important, and the hope is that this text provides some motivation for diving deeper into the field of open quantum systems in general and negative quantum maps in particular.  For example, there is a connection between a lack of complete positivty and non-Markovianity.  This connection is explored in \cite{Rodríguez2012} and \cite{Mazzola2012}), and it may provide deeper insight into some of the problems presented in this text.  The study of non-Markovian quantum stochastic processes is, however, a vast topic on its own, and its intersection with complete positivity cannot be done justice in the space provided here.  I encourage the interested reader to explore the references given above.  The bibliography at the end of this work is meant to not only support statements made in the text, but also to help guide the interested reader further into the field.

\hfill{\em  James M. McCracken}

\hfill{\em May 2014}




\newpage
\chapter*{Aknowledgments}
Cesar Rodriguez-Rosario has helped me many times as I wrote this manuscript and cannot be thanked enough.  I am indebted to him for everything from the content to the form of this work.  I would also like to thank James Troupe for his help with the editing, and Marco Lanzagorta for the opportunity to write this work.

\newpage

