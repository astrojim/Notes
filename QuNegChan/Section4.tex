\chapter{Conclusions}

Complete positivity is not always a  physical requirement of quantum operations, and not completely positive (i.e.\ negative) channels need to be studied to better understand how modern quantum information theory concepts can be implemented and tested in the lab.  A complete theory of quantum information must necessarily include an understanding of negative channels.  Negative channels can be created in the lab, and the degree to which a channel is not completely positive (i.e.\ the negativity) can be measured in quantum process tomography experiments.  All of these points have been made, but it should be emphasized that the dropping of the complete positivity requirement for quantum channels is somewhat expected within the community.  Many different avenues of research are pointing to the abandonment of complete positivity as a physical requirement for quantum operations.

\section{Complete Positivity in Recent Years}

In 2007 Alicki and Lendi argued \cite{Alicki2007}:

``Complete positivity (CP) has experienced its absolute breakthrough in the wide and very active fields of quantum information, particularly in quantum computing. It has been recognized that the CP requirement must unavoidably be imposed on operations affecting only one component of entangled systems, since otherwise artificial and unphysical correlations or ill-defined states may emerge.''

This quote refers to work of Sewell \cite{Sewell2002} and Alicki et al.\ \cite{Alicki2001} which follow the total domain argument\footnote{See Section \ref{sec:physmot}.} for complete positivity as a physical requirement.  This remark comes from a paper entitled ``Recent Developments'' in \cite{Alicki2007}.  Sudarshen et al.\ have pointed out the problems of complete positivity assumptions repeatedly \cite{Sudarshan2005} \cite{Rodriguez2008A} \cite{Sudarshan1978} \cite{Sudarshan1976}, even pointing out the restrictions it places on the $T_1$ and $T_2$ decay rates as far back as 1978.  Yet, as this recent quote shows, negative channels are not very popular in at least part of the community.  The community as a whole, however, appears to be slowly embracing the idea of negative channels.

In 2006, Terno delivered a talk entitled ``Non-Completely Positive Maps in Physics'' \cite{Terno2006} in which he cites the work of \v{S}telmachovi\v{c} and Bu\v{z}ek \cite{Buzek2001} and \.{Z}yczkowski and Bengtsson.  The former do not comment on complete positivity as a physical requirement, but the latter defend their work on non-completely positive maps by arguing: ``Even though positive, but not completely positive, maps cannot be realized in the laboratory, they are of a great theoretical importance,$\ldots$'' \cite{Zyczkowski2004}.  This quote seems to imply that the authors believe negative channels are not physical, but important nonetheless.  Terno also cites Sudarshen et al.\ \cite{Jordan2004} who states rather straightforwardly, ``In the light of understanding gained here, it is easy to see the errors in arguments that a map describing the evolution of an open quantum system {\em has to be completely positive}.''\footnote{Emphasis in original.}  

In 2009, Rodriguez-Rosario wrote a thesis in which he introduces an example of a negative channel and discusses the lack of complete positivity in light of non-Markovian dynamics \cite{Rodriguez2008}.  That same year Wood, wrote a thesis titled ``Non-completely positive maps: properties and applications'' \cite{Wood2009} in which he discusses the experimental evidence of negative channels seen by Havel et al.\ (see Section \ref{sec:Havel}) and whether or not such observations can be explained as statistical errors in the tomography process.  This work includes a discussion of methods with which ``$\ldots$one can distinguish between the distributions of CP [completely positive] and true non-CP processes with a high degree of accuracy'' where the author is concerned about statistical errors in the tomography process leading to observations of a negative channel that should theoretically be completely positive.  But, notice that this quote assumes the existence of ``true non-CP'' quantum processes.  The thesis of Shabani \cite{Shabani2009} was also written in 2009 and extends the concepts of fault tolerant quantum computing to non-completely positive maps, explicitly pointing out that the product state argument\footnote{See Section \ref{sec:physmot}.} is not applicable in most quantum error correction scenarios.  

The push to experimentally realize the ideas of quantum information theory has led to some direct conflicts with complete positivity.  Negative channels are commonly observed in quantum process tomography experiments and a few authors (like the ones mentioned above) want to study why this happens.  The theoretical study of quantum information concepts in ``realistic'' scenarios leads to conclusions which can only be addressed with a theory beyond completely positive maps (as pointed out by Shabani).  The limitations of completely positive maps become more pronounced as quantum information theory seeks to explain more and more of the natural world.  For example, consider this quote from Fleming and Hu in 2011 \cite{Fleming2012}:

``In summary, completely-positive maps are much less useful outside of the Markovian regime, as one rarely has the all-time maps. `All-time' here meaning that the maps must describe all times wherein there may not be any correlation to anything. More precisely, such a map would have to describe the entire universe from its very birth. Typically and empirically, one usually has information only pertaining to the two-time maps of some limited set of states. These maps are not completely positive in the non-Markovian regime and not much is known about them.'' 

Here the authors are describing the limitations of maps with vanishing negativity in a discussion of the Markov assumption in open quantum systems.  Complete positivity describes an idealized theoretical model of a quantum operation useful in its mathematical simplicity but limited in its experimental applicability.   

\section{Closing Remarks}
Our motivation is nicely summed up in a quote by Yuen in his discussion of the uncertainty principle \cite{Yuen2005}:

``It seems to me that current quantum information science and technology also suffers from a number of inadequacies in its foundation. In particular, many mathematical models that have been extensively analyzed are not sufficiently connected to physical and conceptual considerations on realistic experimental situations that would allow one to draw useful conclusions for real applications .$\ldots$ It is important to remember that physicists and engineers need quantitative theories, not just qualitative or asymptotic ones, to build real systems.''

A point needs to be belabored here about the study of negative channels.  Complete positivity is often considered a ``special case'' even by proponents of negative channels.  Extending quantum information theory to include negative channels is often seen as a generalization of the theory.  Completely positive channels are a ``special case'' of negative channels, but this language should be taken only to mean that they are a very specific case.  It should not be taken to mean that negative channels are not as common as completely positive channels, or that most physical situations that arise in the lab are adequately described with completely positive maps.  This attitude is precisely what we were referencing when we mentioned in the previous chapter that the lack of acceptance of negative channels is probably due more to indifference than anything else.  

The complete positivity assumption is not always physically reasonable.  This point was made clear in the Rabi channel examples (see Sec.\ \ref{sec:rabi}), where it was shown that the complete positivity requirement is only applicable to a very specific set of parameter values for the reduced system.  An a priori assumption of complete positivity is an a priori assumption that the physical parameters of the channel have very specific values.  We have shown that the energy eigenbasis evolution of a qubit channel is almost always negative if the composite Hamiltonian is time independent and the bath consists of a single qubit that is initially prepared as a Hadamard rotation of the initial state of the reduced system.  This result implies that if completely positivity is required, then most mathematical constructions of such channels are not physically reasonable.  Perhaps the strangest conclusion of this line of logic is that given a fixed set of parameters in the Hamiltonian, these channels are only ``physical'' at certain, specific points in time.

There are, of course, many situations when the complete positivity assumption is reasonable.  For example, if the initial correlation is assumed to be a product state with a fixed bath, or if the composite dynamics are assumed to have local unitary form, then the reduced dynamics can be assumed to be completely positive.  It should be noted, however, that any conclusions drawn from such assumptions are, as always, limited by those assumptions. 

We have also shown that negative channels can be realized in the lab, and we have suggested ``controlled bath'' experiments that would allow models of negative channels to be verified.  We have argued that experimental observations of negative channels have already occurred, but performing the suggested ``controlled bath'' experiments would serve as a very strong argument against complete positivity as a physical requirement.  If negative channels can be observed in the lab, then complete positivity cannot be a requirement of all physical channels.  

The ``entanglement witness'' argument is a popular argument in favor of complete positivity.  For example, see \cite{Preskill2004,Royer1996}.  The argument relies on a world view that can be depicted as
\begin{equation}
\begin{tikzpicture}[thick]
    \draw[thick, rounded corners, drop shadow={shadow scale=1.005}, fill=red!30!blue!10!white] (-8,-2) rectangle (6.0,2);
    \draw (-7.5,1.5) node[scale=2] {$\mathfrak{U}$};
    \draw[thick, fill=green!50!white] (-6.5,0) circle (0.8cm);
    \draw (-6.5,0) node[scale=1] {$B$};
    \draw[thick, fill=red!50!white] (-1.0,0) circle (0.8cm);
    \draw (-1.0,0) node[scale=1] {$S$};
    \draw[thick, fill=yellow!50!white] (4.5,0) circle (0.8cm);
    \draw (4.5,0) node[scale=1] {$W$};
    \draw[thick] (-3.75,0) ellipse (4.0cm and 1.5cm);
    \draw (-3.75,1.0) node[scale=1.25] {Composite System};
    \draw[thick] (1.75,0) ellipse (4.0cm and 1.5cm);
    \draw (1.75,-1.0) node[scale=1.25] {Entanglement};
    %
    %
\end{tikzpicture}
\end{equation}
where $\mathfrak{U}$ is the universe, $B$ is the bath, $S$ is the reduced system, and $W$ is the entanglement witness.  The entanglement witness is part of neither the reduced system nor the bath, but it is still expected that the system consisting of both $S$ and $W$ (the system ``$SW$'') must be represented by a valid density matrix.  Why is $W$ not a part of $B$?  It might be claimed that $W$ is not intended to be a physical object.  Instead, it might be argued, $W$ should just be considered a ``witness'' to the physics.  Notice, however, that the very claim of complete positivity relies on $W$ being a physical object.  

The requirement of complete positivity is a requirement of a valid (specifically positive) density matrix representation for $SW$.  If $W$ is not physical, why should $SW$ be expected to have a valid density matrix representation?  At most, it might be expected that $(SW)^\flat = S$, but notice that this is a requirement of consistency, not complete positivity.  The system $W$ has no direct correlation with the bath, but it imposes restrictions on the reduced dynamics because, as is often argued, it can never be known if the system $W$ exists or not and $S$ must always evolve into another density matrix.  It has already been pointed out that negative channels can have positivity domains that cover all of the reduced system space.  So, complete positivity is not required to insure the positivity of $S$.  

Notice also that if $B$ were maximally entangled with $S$, then the evolution of $S$ would be unaffected only if the composite dynamics were in local unitary form.  This statement would be true independently of the experimenter's knowledge of the existence of $B$.  It is implied that the reduced dynamics must be completely positive because the existence of $W$ cannot influence the channel.  The existence of $B$, however, must influence the channel.  Such logic is confusing and appears contrived.  Complete positivity is a nice property, but philosophical arguments about entanglement witnesses do not provide the sound physical motivation required to impose it on all channels as an a priori assumption.  These points were argued at the beginning of Chapter 2, but hopefully returning to them in this conclusion, after the introduction and analysis of several different negative channel examples, makes the point a little clearer. 

As a final note, it should be recognized that most of the discussion we have presented has implications far beyond quantum information theory.  Everything here was presented in the framework of quantum information theory but the conclusions extend to any open quantum system.  Complete positivity is an assumption in the field of open quantum systems and that field has been applied to everything from non-equilibrium thermodynamics to high energy physics \cite{Breuer2007}.  For example, the time inversion operator is often considered to not be physical because it is known to not be completely positive \cite{Busch1990}.    

The proposed experimental measurements of negativity we presented are not the first proposed experimental tests of complete positivity.  In the late 1990s, it was pointed out that complete positivity puts bounds on experimental parameters in both neutral kaon experiments \cite{Benatti1996} \cite{Benatti1997} \cite{Benatti1998} and neutron interferometry experiments \cite{Benatti1999}.  Experiments with neutral kaon systems were proposed as a way to test complete positivity \cite{Benatti1997a}.  The experiments we propose are much simpler (both theoretically and experimentally) and involve the concept of negativity (i.e.\ quantification of the lack of complete positivity), but the proposed neutral kaon system experiment illustrates the far reach of the complete positivity assumption.  Complete positivity has observable effects, and, if required, it limits the physical processes that are possible.  This is a limitation felt in all of theoretical physics, not just quantum information.   

There are many open questions in the study of negative channels.  Our intent is to point out that the assumption of complete positivity is not always reasonable.  Once these ideas are confirmed by experiment (e.g.\ as proposed in Section \ref{sec:proposedexp}), then the study of negative channels can be used to better understand how to implement many of the currently proposed quantum technologies, and it might even lead to new proposed technologies considered impossible under the assumption of complete positivity.

