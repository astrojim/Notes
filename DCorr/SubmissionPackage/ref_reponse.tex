\documentclass[]{article}

%opening
\title{}
\author{}

\begin{document}

\maketitle

We have edited our manuscript, and we believe we have addressed all the criticisms raised by the referee.  We appreciate the referee taking the time to review our work.  Our responses to each critique are below:

Major Point 1: ``It is not clear why their first two models (discrete models) have noise term but their last model doesn't have a noise term. It looks like the noise term was introduced in those two models to work their benefit (i.e.) to show the limitation of CCM. A clear justification for the noise term should be provided.''

Response 1: In response to this point, text has been added to the new manuscript at the end of Section 3A.  In summary, the failure of the CCM algorithm in not due to noise alone.  The following text has been added to Section 3A, (paragraph 4) in the new manuscript addresses these points: ``These results do not imply that the primary failure of CCM causality is not being reliable at certain signal-to-noise ratios.  The Eqn.\ \ref{eq:linearex} system with no noise, i.e.\ $B=0$, leads to $\Delta=2.5\times 10^{-3}$, which does not agree with intuition.  However, Fig.\ \ref{fig:linearex1} shows CCM causality for this example does agree with intuition for certain noise levels.  The key idea is that the agreement of CCM causality with intuition depends on the system parameters.  The value of $\Delta$ is strongly dependent on the system parameters even in noise-free systems (see Section \ref{sec:rlcirc}). ''

(The noise term was used for the examples of Section 3A and 3B in the to explore the dependence of the CCM algorithm on the system parameters.  The CCM algorithm does fail without the noise term, but our objective was to explore the failure of the CCM algorithm in more depth.  This reason is also why there was no noise term in the example of Section 3C.  Here the CCM algorithm is still strongly dependent on the system parameters, but now none of the parameters are noise terms.), 

Major Point 2: ``Their presentation of results was poor. It is not clear why more than one coupling coefficient have been used in their first two models. No justification was provided for those coefficients. It looks like they were introduced so the results can be presented as contour plots.  Can't the limitation of CCM be shown just with one coupling parameter in the first two models? The results would also look simpler. There won't be a need to use a contour diagram unnecessarily in this case.''

Response 2: In an early draft, we used a single parameter in the examples but later decided the two-parameter space better illustrated the unexpected behavior of the CCM algorithm (see Response 1).  However, we appreciate that this choice had a potential to make the presentation more difficult to follow.  We have compromised by changing the example of Section 3A to eliminate one of the parameters.  Figure 2 (and 7) have been changed correspondingly, but our conclusions remain the same.

Major Point 3: ``Examples from physical/physiological/meteorological system would strengthen their claims.''

Response 3:  We have interpreted this comment in two different ways, (1) as a lack of testing the algorithms with empirical data and (2) as a lack of testing the algorithm with complex physical dynamical systems like the Lorentz system.  To address (1), we have added Section 5, which applies both the CCM and the PAI algorithms to altitude/temperature data.  To address (2), we have added the following text to the beginning of Section 3: ``The usefulness of the CCM algorithm in identifying drivers among sets of time series can be explored by using simple example systems.  The benefit to this approach is that the driver is known beforehand.  Thus, CCM causality can be validated by comparison against intuition in simple systems where intuition can be trusted.  Intuitively identifying the driving signal in the coupled logistic map presented in \cite{Sugihara2012} is difficult due to the complexity of the system.  Commonly used chaotic dynamic systems such as the Lorentz system present similar difficulties.  Thus, the examples here are intended to explore (and validate) the CCM method by comparing the results to expected conclusions.''  We had previously tested the CCM algorithm using the Lorentz system but found it difficult to interpret and validate the results.  This difficulty lead us to consider simpler systems in which the CCM algorithm could be better validated as a tool for causal inference.  Complex dynamical systems like the Lorentz system are less useful for this task than simple  systems with clear driving signals (or empirical data sets with known causal relationships).   The current literature contains many examples of the use of CCM on complex physical systems.  The work presented here was a direct result of difficulties that we had when attempting to interpret CCM results on complex physical systems.  We decided that applying the CCM or PAI algorithms to dynamical systems and empirical data without known causal relationships is better omitted from the results presented in this manuscript, for which the focus is on a better understanding the CCM algorithm.

Minor Point 1: ``There are several methods to look at the driver-response relationships (partial directed coherence and similar frequency domain methods).  This being the case, why would one want to use the work proposed here  (i.e.) a good motivation is lacking.''

Response 4:  In response to this, we have added text to the introduction that better explains both time series causality and the original motivation for the CCM methods.  The new text reads: ``To date, most techniques for ``causal inference'' in time series data fall into three broad categories, those related to transfer entropy, those related to Granger causality, and those related to lagged cross-correlation.  Transfer entropy (introduced in \cite{Schreiber2000}) and Granger causality (introduced in \cite{granger1969}) are known to be equivalent under certain conditions \cite{Barnett2009}.  In this article, we investigate a casual inference technique, called Convergent Cross-Mapping (CCM), that was recently introduced by Sugihara {\em et al.\ } \cite{Sugihara2012}.  (Currently, there is no evidence that CCM is related to either transfer entropy or Granger causality.)

Statements of causality beyond simple correlation that are derived from time series data alone fall under the broad term of ``time series causality''.  Most known time series causality measures have well-documented shortcomings.  Granger causality (and its extensions) are model-based approaches that depend on, among other things, the validity of the model \cite{Kaminski2001,Granger1980} and separability \cite{Sugihara2012}.  Transfer entropy, mutual information, and other information theoretic techniques rely on estimating entropies from empirical data, which may involve the computationally difficult task of estimating transition probability densities, and generally require large amounts of data \cite{Kaiser2002,Schindler2007}.  Lagged cross-correlation techniques can be unreliable in the presence of strongly autocorrelated data \cite{box2013}, and partial directed coherence (which is related to lagged cross-correlation techniques) can lead to spurious conclusions given realistic data \cite{pascual2014}.  The results documented here show that the recently introduced CCM techniques have similar shortcomings.  

CCM is described as a technique that can be used to identify ``causality'' between time series and is intended to be useful in situations where Granger causality is known to be invalid (i.e.\ in dynamic systems that are ``nonseparable'' \cite{Sugihara2012}).  CCM is also intended to useful in determining causality in systems that experience ``mirage'' correlations \cite{Sugihara2012} (i.e.\ correlations that are a function of time).  Thus, CCM is introduced as a time series causality technique that specifically addresses some of the shortcomings discussed in the previous paragraph.  The authors state that CCM is a ``necessary condition for causation''.  It is well known that Granger causality is not causality as it is typically understood in physics \cite{Granger1980,liu2012,Roberts1985}.  Whether or not a similar conclusion can be drawn regarding CCM causality is currently open question.''

Minor Point 2:  ``x-axis label is missing for Figure 2b.''

Response 5:  Figure 2b has been replaced by a new plot (see Response 2).


\end{document}
