\documentclass[]{article}

\usepackage[english]{babel}
\usepackage[utf8x]{inputenc}
\usepackage{amsmath}
\usepackage{graphicx}
\usepackage[colorinlistoftodos]{todonotes}

\title{Directed Correlation as Evidence for Pairwise Asymmetry in Time Series Data Sets}
\author{Weigel,McCracken}

\begin{document}
\maketitle

\begin{abstract}
Directed correlation is defined using the Convergent Cross Mapping (CCM) technique introduced by Sugihara {\em et.\ al.\ } in \cite{}.  It is shown that CCM correlations do not, in general, agree with intuitive concepts of ``driving'' and ``response'', and as such, relationships among CCM correlations should not be considered indicative of causality.  CCM correlations can, however, be used to discover asymmetrical prediction potential between pairs of time series data.  We define a 2-vector called the ``directed correlation'' and present several examples of its use and interpretation.  The sensitivity of CCM correlations (and consequently, directed correlations) on embedding dimensions and lag times will be discussed and mitigation will be presented.
\end{abstract}

\section{Introduction}

\todo[inline, color=green!40]{TBD}

\end{document}